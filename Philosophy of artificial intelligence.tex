\documentclass[10pts]{article}
\begin{document}
\title{Philosophy of artificial intelligence}
\maketitle
\section{ Ethics of artificial intelligence}
The philosophy of artificial intelligence is a branch of the philosophy of technology that explores artificial intelligence and its implications for knowledge and understanding of intelligence, ethics, consciousness, epistemology, and free will.Furthermore, the technology is concerned with the creation of artificial animals or artificial people (or, at least, artificial creatures; see artificial life) so the discipline is of considerable interest to philosophers. some of the propositions in the philosophy of AI are:"
\section{Turing's "polite convention"}
Turing's "polite convention": If a machine behaves as intelligently as a human being, then it is as intelligent as a human being.\\
\section{The Dartmouth proposal}
The Dartmouth proposal: "Every aspect of learning or any other feature of intelligence can be so precisely described that a machine can be made to simulate it."\\
\section{Allen Newell and Herbert A. Simon's physical symbol system hypothesis}
Allen Newell and Herbert A. Simon's physical symbol system hypothesis: "A physical symbol system has the necessary and sufficient means of general intelligent action."\\
\section{John Searle's strong AI hypothesis}
John Searle's strong AI hypothesis: "The appropriately programmed computer with the right inputs and outputs would thereby have a mind in exactly the same sense human beings have minds."\\
\section{Hobbes' mechanism}
Hobbes' mechanism: "For 'reason' ... is nothing but 'reckoning,' that is adding and subtracting, of the consequences of general names agreed upon for the 'marking' and 'signifying' of our thoughts..."\\
\section{Intelligence}
Is it possible to create a machine that can solve all the problems humans solve using their intelligence? This question defines the scope of what machines could do in the future and guides the direction of AI research. It only concerns the behavior of machines and ignores the issues of interest to psychologists, cognitive scientists and philosophers; to answer this question, it does not matter whether a machine is really thinking (as a person thinks) or is just acting like it is thinking.\\
\section{Intelligent agent}
An "agent" is something which perceives and acts in an environment. A "performance measure" defines what counts as success for the agent. "If an agent acts so as to maximize the expected value of a performance measure based on past experience and knowledge then it is intelligent."\\
\section{Human thinking is symbol processing}
In 1963, Allen Newell and Herbert A. Simon proposed that "symbol manipulation" was the essence of both human and machine intelligence. they wrote;"A physical symbol system has the necessary and sufficient means of general intelligent action." and "The mind can be viewed as a device operating on bits of information according to formal rules."\\
\section{Arguments against symbol processing}
These arguments show that human thinking does not consist (solely) of high level symbol manipulation. They do not show that artificial intelligence is impossible, only that more than symbol processing is required.
\end{document}