\documentclass[10pts]{article}

\begin{document}

\title{Gödel's incompleteness theorems}

\maketitle


Gödel's incompleteness theorems are two theorems of mathematical logic that are concerned with the limits of provability in formal axiomatic theories. These results,are published by Kurt Gödel in 1931, are important both in mathematical logic and in the philosophy of mathematics. The theorems are widely, but not universally, interpreted as showing that Hilbert's program to find a complete and consistent set of axioms for all mathematics is impossible\\.

The first incompleteness theorem states that no consistent system of axioms whose theorems can be listed by an effective procedure (i.e., an algorithm) is capable of proving all truths about the arithmetic of natural numbers. For any such consistent formal system, there will always be statements about natural numbers that are true, but that are unprovable within the system. The second incompleteness theorem, an extension of the first, shows that the system cannot demonstrate its own consistency.\\

Employing a diagonal argument, Gödel's incompleteness theorems were the first of several closely related theorems on the limitations of formal systems. They were followed by Tarski's undefinability theorem on the formal undefinability of truth, Church's proof that Hilbert's Entscheidungsproblem is unsolvable, and Turing's theorem that there is no algorithm to solve the halting problem.\\



\end{document}
