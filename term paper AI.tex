\documentclass{article}
\begin{document}
\title{Deep learning in medical diagnosis }
\author{Alan Biju}


\maketitle


\section*{Abstract}

Deep learning (DL) is an Artificial neural network-driven framework with multiple levels of representation for which non-linear modules combined in such a way that the levels of representation can be enhanced from lower to a much abstract level. Though DL is used widely in almost every field, it has largely brought a breakthrough in biological sciences as it is used in disease diagnosis and clinical trials. DL can be clubbed with machine learning, but at times both are used individually as well. DL seems to be a better platform than machine learning as the former does not require an intermediate feature extraction and works well with larger datasets. DL is one of the most discussed fields among scientists and researchers these days for diagnosing and solving various biological problems. However, deep learning models need some improvisation and experimental validations to be more productive 


\section*{Introduction}
Symptoms can be completely different, but these would be cases when the disease is entirely asymptomatic. Even if specialists have diagnosis problems, diagnostic errors are recognized as the most common and harmful medical errors, since 12 to 18 million Americans face some misdiagnosis.

It is important to note that physician work is usually not a direct cause of diagnostic errors. Research shows that a number of factors are responsible for misdiagnosis,


\end{document}