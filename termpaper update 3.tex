\documentclass{article}
\begin{document}
\title{Deep learning in medical diagnosis }
\author{Alan Biju}


\maketitle


\section*{Abstract}

disease diagnosis is one of the most important procedure which can improve the effectiveness of treatments and it requires a much-experienced physician. But it is very difficult to make out each and every symptom accurately as there are many diseases which are asymptomatic. Also, diagnostic errors are most common harmful medical error. So, in this paper we will see how artificial intelligent, deep learning can play a crucial role in medical diagnosis. And we will see in detail about what is deep learning. 



\section*{Introduction}


Deep learning (DL) is an Artificial neural network-driven framework with multiple levels of representation for which non-linear modules combined in such a way that the levels of representation can be enhanced from lower to a much abstract level. The basic idea of deep learning is to mimic human brain. Even though deep learning can be helpful in many fields, its introduction in biological science has made a deep impact in helping physicians in detecting and diagnosis of many diseases, and is also useful in medical imaging etc. When it comes to artificial intelligents we can say as deep learning is a subset of machine learning. Even though both can be used to detect or diagnose disease, deep learning seems to be a better platform than machine learning as deep learning requires the need of human in differentiating the data. While DL does not require feature extractions and will work best with large datasets and there by removes the dependency on humans. 


 \section* {How deep learning works }

Deep learning is considered as artificial neural network which mimic the human brain through a combination of data inputs, weights, and bias. These three works together to attain classification, describe objects within the data. Deep neural networks consist of multiple layers of interconnected nodes, each building upon the previous layer to refine and optimize the prediction or categorization. This progression of computations through the network is called forward propagation. The input and output layers of a deep neural network are called visible layers. The input layer is where the deep learning model ingests the data for processing, and the output layer is where the final prediction or classification is made. 

Another process called backpropagation uses algorithms, like gradient descent, to calculate errors in predictions and then adjusts the weights and biases of the function by moving backwards through the layers in an effort to train the model. Together, forward propagation and backpropagation allow a neural network to make predictions and correct for any errors accordingly. Over time, the algorithm becomes gradually more accurate. These discussed over here are some simple nural network however these can be complex like Convolutional neural networks (CNNs) 


\section*{Convolutional neural networks (CNN)} 


In neural networks, CNN is a unique family of deep learning models. CNN is a major artificial visual network for the identification of medical image patterns. The family of CNN primarily emerges from the information of the animal visual cortex. The major problem within a fully connected feed-forward neural network is that even for shallow architectures, the number of neurons may be very high, which makes them impractical to apply to image applications. The CNN is a method for reducing the number of parameters, allows a network to be deeper with fewer parameters., CNNs is an efficient technique for detecting features of an object and achieving good classification performance. There are drawbacks to CNNs, which are that unique relationships, size, perspective, and orientation of features are not taken into account. To overcome the loss of information in CNNs by pooling operation Capsule Networks (CapsNet) are used to obtain spatial information and most significant features. The special type of neurons, called capsules, can detect efficiently distinct information. The capsule network consists of four main components that are matrix multiplication, Scalar weighting of the input, dynamic routing algorithm, and squashing function 

 
\section*{APPLICATION }


Even though there are many applications in many fields in this paper we are considering medical applications. But this requires a lot of labelled data for its neural network training. But in medical field its difficult to obtain a vide verity of labeled data. This issue is eased by a strategy called transfer learning. Two exchange learning approaches are well known and generally applied that are fixed component extractors and calibrating a pre-prepared organization. In the classification process, the deep learning models are used to classify images into two or more classes. In the detection process, Deep learning models have the function of identifying tumors and organs in medical images. In the segmentation task, deep learning models try to segment the region of interest in medical images for processing. basically, saying this deep learning technique can be very helpful in early diagnosis of any tumor and in medical image processing. 

According to WHO many people have been associated with breast cancer over the years so early scanning is required but it become difficult for doctors as theere are many cases, so the introduction of deep learning can help the doctors in scanning of tumors in a better speed. By training the agent with many data sets deep learning can be helpful in diagnosis of many disorders like arrythmia, lung cancer etc. 


\section*{Limitations and challenges }



 As we saw there are many applications for deep learning algorithm which help in image diagnosis but behind all this there are some limitations. The main limitation is inconsistency of data, The non-standardized acquisition of medical images is another limitation in medical image analysis. 

Medical datasets are really difficult to obtain and very less when compared to other data sets, as all humans don’t like to share their medical reports to others. In Oder to build a DLA, it requires a large amount of annotated data. Annotating medical images becomes a major task as Labeling medical images requires a highly skilled radiologist and a deep knowledge. Therefore, it's a time-consuming to annotate adequate medical data. Semi-supervised learning could be implemented to make combined use of the existing labeled data and vast unlabeled data to alleviate the issue of “limited labeled data”. Another way to resolve the issue of “data scarcity” is to develop few-shot learning algorithms using a considerably smaller amount of data.


\section*{Result  }


The main inference that I understood after writing this paper is that within few years there are possibilities to overcome each and every obstacle by the use of artificial intelligence in almost every felid, as we see in this deep learning can help in medical diagnosis and help in determine a better treatment with the availability of more labeled data's there will be time where we will fully depend on this algorithm. And also, these deep learning algorithms can help all the expert physicians in confirming their outcome with that DLA, since we use this, we will be able to detect many patients with in a short period of time thus makes a great impact in health care department. 









\end{document}